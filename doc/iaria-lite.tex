\documentclass{ltxdoc}

\CodelineNumbered
\EnableCrossrefs
\CodelineIndex
\RecordChanges
\GetFileInfo{iaria-lite.cls}

\parskip1.0ex
\parindent0.0ex

\begin{document}
\title{\textsf{iaria-lite}\\
 An Unofficial IARIA LaTeX Class (Lite Version)}
\author{Christoph P.\ Neumann \texttt{$<$cyberpetaneuron@gmail.com$>$}}
\date{Version 0.7}
\maketitle
\PrintChanges

\begin{abstract}
The \texttt{iaria-lite}--class provides a convenient environment for writing IARIA scholary publications.
(The lite version of the class file does not implement IARIA specifications for citation style,
because this would require to make presumption about technological stacks like biblatex/biber).
The \texttt{iaria-lite}--class should be compatible with with all latex distributions.
\end{abstract}

\section{Installation}

The \texttt{zip} or \texttt{tar.gz} file comes with a \texttt{iaria-lite.ins}
and a \texttt{iaria-lite.dtx} file included which contains the \LaTeX\ stuff.

To extract the class files call:

\begin{verbatim}
  $ latex iaria-lite.ins
\end{verbatim}

This call will extract all \LaTeX\ specific files to the current
directory. You can either use the files for a single
cv project or you can integrate the files into your \TeX\ installation.

If you just want to use \textsf{iaria-lite} for a single curriculum vitae
project, the simplest way is just to copy the generated files to the
folder of the project.

If you want to integrate \textsf{iaria-lite} into  your \TeX\ installation,
create a directory \texttt{tex/latex/iaria-lite} beneath your \TeX\ installation
(e.g.~beneath \texttt{/usr/share/texmf}) and copy all files from the
current directory there. Now call:

\begin{verbatim}
  $ mktexlsr
\end{verbatim}

to update the file--cache of \LaTeX.

Hint: The \textsf{iaria-lite} distribution contains a sample docstrip configuration
in \texttt{docstrip.cfg} via which files can be distributed
automatically to their correct positions inside a \LaTeX\ installation.
Feel free to adapt this file to your environment and afterwards call
\texttt{latex iaria-lite.ins} to install the package to its right place.

\section{Templates}

For a quick start the \textsf{iaria-lite} distribution contains document
templates.
The templates can be found in the \texttt{iaria-lite-example-ctan.zip} file.

Especially beginners should NOT start from scatch! Use the example file!
The iaria-lite package assumes that the LaTeX user has former experience with IEEEtran!
For example, the formatting of the author block is not documented here, please refer to the IEEEtran documentantion.
For shared affiliations, there is a formatting option called ``alternate long format'', which you can find in the well-known IEEEtran\_HOWTO.pdf;
it is also the superior option for IARIA papers with shared affiliations.

BEWARE: If you are completely new to LaTeX, then do NOT use iaria-lite class for your first LaTeX experiment, but please use the IARIA Word template.
The same advice is true when you have no former IEEEtran experience and when you are not willing to acquire IEEEtran knowledge on your own.

DISCLAIMER: Using this package still requires that you additionally MUST read the
IARIA formatting guide (www.iaria.org/formatting.doc)
as well as the IARIA editorial rules (www.iaria.org/editorialrules.html).
Simply using this package does not relieve you of your responsibility!
This applies to all aspects: IARIA style and LaTeX/BibTeX know how (like interfering packages) as well as IEEEtran implications (like conference mode, author block, shared affiliation, or cite compression).
In case of doubt, you are required to understand the content of the iaria-lite.cls file, because before this this community-provided you had to make all these adjustments yourself.

\section{Documentclass}

\DescribeMacro{documentclass iaria-lite}
This package provides the documentclass \texttt{iaria-lite}. The documentclass
supports the following options:

\begin{itemize}
\item |conference| Passed to IEEEtran (i.\,a., it suppresses page numbers)
\item |a4paper|    Passed to IEEEtran
\item |subfig|     Loads subfig package with IARIA style settings
\item |subcaption| Loads subcaption package with IARIA style settings
\item |flushend|   Activate flushend package (compatible with arXiv build process)
\item |pbalance|   Activate pbalance package (incompatible with arXiv build process)
\end{itemize}

BEWARE: Usually, you will/must provide IEEEtran option ``conference'' to the iaria package, both for IARIA conferences AND for IARIA journals! Because for IARIA, the formatting rules are the same for conferences and journals -- in contrast to IEEE. Its conference style is the base for IARIA formatting.
The example file (see above) prominently applies this option.

A Remark about two-column document balancing on the last page:
The flushend package is recommended, because it works within the arXiv automated build process.
However, flushend has a major incampatibility with package lineno, which is, e.\,g., transitively loaded by package mindflow.
Thus, in case when flushend does not have any effect, check whether one of your packages loads lineno.
Usually you are stuck with your packages and, thus, will instead be forced to switch from flushend to pbalance.
Please be aware that pbalance works great and has high compatibility, but unfortunately it will not have any effect within the arXiv automated build process.
In case of both, a somehow needed lineno package and an intended arXiv upload, I recommend to do without two-column balancing on the last page and just to stay away from both flushend and pbalance, in order to ensure that your paper is layouted identically in IARIA submission and arXiv upload.
I hope this remark proves helpful, it took me some nerve to find out.

There is also another important difference between flushend and pbalance: the handling of footnotes on the last page.
Flushend provides a decent result, but the result of pbalance is incomprehensible.
I strongly recommend avoiding footnotes on the last page.
If you absolutely need footnotes on the last page, consider staying away from two-column balancing on the last page.

About subfigures:
Both well-known packages subfig and subcaption can be used.
However, they are not compatible with each other and we can either load one or the other.
Both require dedicated style settings to be compatible with IARIA formatting rules.
Thus, the document class provides options to load them for you with correct settings.

In order, to configure table/figure captions correctly for IARIA requirements, the package caption is used.
There would be other options to achieve this, however, the babel package is disruptive to the table label -- using package caption provides stability to the label configuration against babel intervention.
There are minor downsides of using caption package:
it results in a warning about unknown document class that cannot be suppressed, and if you do not use any tables or figures, caption package also results in a warning about ``unused captionsetup'' that cannot be suppressed, either.

\section{Requirements}

We instrument several other \LaTeX\ packages for different purposes,
which must be available under your installation.

\begin{itemize}
\item IEEEtran
\item extdash
\item floatrow
\item flushend
\item graphicx
\item hyperref
\item orcidlink
\item pbalance
\item subcaption
\item subfig
\item times
\item url
\item xcolor
\item xpatch
\end{itemize}

\end{document}